\section{Tasks Status} % Seções são adicionadas para organizar sua apresentação em blocos discretos, todas as seções e subseções são automaticamente exibidas no índice como uma visão geral da apresentação, mas NÃO são exibidas como slides separados.

%----------------------------------------------------------------------------------------

\begin{frame}
  \frametitle{Tasks Status}
    \begin{table}[h]
        \centering
        \begin{tabular}{|l|c|c|}
        \hline
        \textbf{Task} & \textbf{Status} & \textbf{Link} \\
        \hline
         &  & \\
        \hline
         &  & \\
        \hline
        \end{tabular}
    \end{table}
\end{frame}

%----------------------------------------------------------------------------------------
% \begin{frame}{Uma imagem}
%     \frametitle{Historical Records}
%     In addition, it is possible to retrieve the full search history by running the script directly via GitHub Actions.
%    \begin{figure}[h]
%        \centering
%        \includegraphics[width=0.3\textwidth]{img/github_topic_history.png}
%        \caption{GitHub Actions interface illustrating how the workflow can be executed to retrieve the full search history for a specific topic.}
%        \label{fig:github_topic_history}
%    \end{figure}
% \end{frame}
%----------------------------------------------------------------------------------------

% \begin{frame}
% 	\frametitle{Texto em tópicos}
%      Lorem ipsum dolor sit amet, consectetur adipiscing elit:
%     \begin{itemize}
%         \item Lorem ipsum dolor sit amet.
%         \item Lorem ipsum dolor sit amet.
%     \end{itemize}
	
% \end{frame}


