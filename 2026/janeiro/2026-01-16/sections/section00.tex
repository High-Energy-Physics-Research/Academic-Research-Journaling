\section{Automated arXiv Feed for Literature Monitoring} % Seções são adicionadas para organizar sua apresentação em blocos discretos, todas as seções e subseções são automaticamente exibidas no índice como uma visão geral da apresentação, mas NÃO são exibidas como slides separados.

%----------------------------------------------------------------------------------------

\begin{frame}
    I created the GitHub repository 
\textcolor{blue}{\textbf{\href{https://github.com/Automation-Scripting/literature-alerts-bot}{literature-alerts-bot}}}
to monitor daily new articles from \textbf{arXiv} to a \textbf{Discord} server for the following topics:
\begin{enumerate}
        \item \textbf{hep\_ph}: searches for recent submissions in high-energy physics phenomenology
        \item \textbf{qgp\_ml}: searches for studies applying machine learning techniques to QGP and heavy-ion physics, including emulators, surrogate models, Gaussian processes, and ML-driven modeling of relativistic nucleus-nucleus collisions.
        \item \textbf{qgp\_bayesian}: searches for works on Bayesian inference and Bayesian analysis applied to QGP physics, including parameter estimation, uncertainty quantification, and model calibration in heavy-ion phenomenology.
        \item \textbf{qgp\_dkl}: searches for articles on Gaussian processes and deep kernel learning, including deep Gaussian processes, neural kernels, learned kernels, and hybrid GP-neural network models, with emphasis on modern kernel-learning approaches.
    \end{enumerate}
	
\end{frame}

%----------------------------------------------------------------------------------------
\begin{frame}{Uma imagem}
    \frametitle{Historical Records}
    In addition, it is possible to retrieve the full search history by running the script directly via GitHub Actions.
   \begin{figure}[h]
       \centering
       \includegraphics[width=0.3\textwidth]{img/github_topic_history.png}
       \caption{GitHub Actions interface illustrating how the workflow can be executed to retrieve the full search history for a specific topic.}
       \label{fig:github_topic_history}
   \end{figure}
\end{frame}
%----------------------------------------------------------------------------------------

% \begin{frame}
% 	\frametitle{Texto em tópicos}
%      Lorem ipsum dolor sit amet, consectetur adipiscing elit:
%     \begin{itemize}
%         \item Lorem ipsum dolor sit amet.
%         \item Lorem ipsum dolor sit amet.
%     \end{itemize}
	
% \end{frame}


